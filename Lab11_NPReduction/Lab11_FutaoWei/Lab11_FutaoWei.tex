\documentclass[12pt,a4paper]{article}
\usepackage{ctex}
\usepackage{amsmath,amscd,amsbsy,amssymb,latexsym,url,bm,amsthm}
\usepackage{epsfig,graphicx,subfigure}
\usepackage{enumitem,balance}
\usepackage{wrapfig}
\usepackage{mathrsfs,euscript}
\usepackage[usenames]{xcolor}
\usepackage{hyperref}
\usepackage[vlined,ruled,linesnumbered]{algorithm2e}
\usepackage{array}
\hypersetup{colorlinks=true,linkcolor=black}

\newtheorem{theorem}{Theorem}
\newtheorem{lemma}[theorem]{Lemma}
\newtheorem{proposition}[theorem]{Proposition}
\newtheorem{corollary}[theorem]{Corollary}
\newtheorem{exercise}{Exercise}
\newtheorem*{solution}{Solution}
\newtheorem{definition}{Definition}
\theoremstyle{definition}

\renewcommand{\thefootnote}{\fnsymbol{footnote}}

\newcommand{\postscript}[2]
 {\setlength{\epsfxsize}{#2\hsize}
  \centerline{\epsfbox{#1}}}

\renewcommand{\baselinestretch}{1.0}

\setlength{\oddsidemargin}{-0.365in}
\setlength{\evensidemargin}{-0.365in}
\setlength{\topmargin}{-0.3in}
\setlength{\headheight}{0in}
\setlength{\headsep}{0in}
\setlength{\textheight}{10.1in}
\setlength{\textwidth}{7in}
\makeatletter \renewenvironment{proof}[1][Proof] {\par\pushQED{\qed}\normalfont\topsep6\p@\@plus6\p@\relax\trivlist\item[\hskip\labelsep\bfseries#1\@addpunct{.}]\ignorespaces}{\popQED\endtrivlist\@endpefalse} \makeatother
\makeatletter
\renewenvironment{solution}[1][Solution] {\par\pushQED{\qed}\normalfont\topsep6\p@\@plus6\p@\relax\trivlist\item[\hskip\labelsep\bfseries#1\@addpunct{.}]\ignorespaces}{\popQED\endtrivlist\@endpefalse} \makeatother

\begin{document}
\noindent

%========================================================================
\noindent\framebox[\linewidth]{\shortstack[c]{
\Large{\textbf{Lab11-NP Reduction}}\vspace{1mm}\\
CS214-Algorithm and Complexity, Xiaofeng Gao, Spring 2020.}}
\begin{center}
\footnotesize{\color{red}$*$ If there is any problem, please contact TA Shuodian Yu. }

\footnotesize{\color{blue}$*$ Name: Futao Wei  \quad Student ID: 518021910750 \quad Email: weifutao@sjtu.edu.cn}
\end{center}
\begin{enumerate}
	\item What is the ``certificate'' and ``certifier'' for the following problems?
	\begin{enumerate}
		\item \emph{ZERO-ONE INTEGER PROGRAMMING}: Given an integer $m \times n$ matrix $A$ and an integer $m$-vector $b$, is there an integer $n$-vector $x$ with elements in the set $\{0, 1\}$ such that $Ax \leq b$.
		\item \emph{SET PACKING}: Given a finite set $U$, a positive integer $k$ and several subsets $U_1, U_2, \ldots, U_m$ of $U$, is there $k$ or more subsets which are disjoint with each other?
		\item \emph{STEINER TREE IN GRAPHS}: Given a graph $G=(V,E)$, a weight $w(e)\in Z_0^{+}$ for each $e\in E$, a subset $R \subset V$, and a positive integer bound $B$, is there a subtree of $G$ that includes all the vertices of $R$ and such that the sum of the weights of the edges in the subtree is no more than $B$.
	\end{enumerate}
	\begin{solution}
        \hfill
        \begin{enumerate}
        	\item 
        	\textbf{Certificate:} An integer $n$-vector $x$ with elements in the set $\{0, 1\}$. \\
        	\textbf{Certifier:} An algorithm that checks if $Ax \leq b$.
        	\item 
        	\textbf{Certificate:} A collection $C$ of subsets of $U$ whose size is no less than $k$. \\
        	\textbf{Certifier:} An algorithm that checks if the elements in $C$ are disjoint with each other.
        	\item 
        	\textbf{Certificate:} A subtree of $G$ that includes all the vertices of $R$. \\
        	\textbf{Certifier:} An algorithm that checks if the sum of the weights of the edges in the subtree is no more than $B$.
        	
        \end{enumerate}
    \end{solution}

	\item Algorithm class is a democratic class. Denote class as a finite set $S$ containing every students. Now students decided to raise a student union $S' \subseteq S$ with $|S'|\leq K$ .
	
	As for the members of the union, there are many different opinions. An opinion is a set $S_o\subseteq S$. Note that number of opinions has nothing to do with number of students.
	
	The question is whether there exists such student union $S' \subseteq S$ with $|S'|\leq K$, that $S'$ contains at least one element from each opinion. We call this problem \emph{ELECTION} problem, prove that it is NP-complete.
	\begin{proof}
		First, \emph{ELECTION} problem is NP, since we can find a poly-time certifier. What the certifier does is to check the students in $S'$ one by one to determine whether there's at least one element from each opinion group. \\
		Next, we intend to establish NP-completeness of \emph{ELECTION} problem by proving $\emph{SET COVER} \leq _P \emph{ELECTION}$. \\
		Instance of \emph{SC}: A set $U$; a collection of $m$ subsets of $U$ ($S_1, S_2, \dots , S_m$); an integer $K$ \\
		Instance of \emph{ELECTION}: The same set $U$, whose elements denote all kinds of opinions; the same collection of $m$ subsets of $U$ ($S_1, S_2, \dots , S_m$), $S_i$ denotes the opinions the $i ^{\text{th}}$ student holds; the same integer $K$; a set $S$ of all the students \\
		(Note: The instance of \emph{ELECTION} above can be transformed to the original form) \\
		Then we prove that there exists a set cover of size $\leq K$ iff there exists a student union $S' \subseteq S$ with $|S'|\leq K$. \\
		$\implies$: \\
		If there exists a set cover of size $\leq K$, namely, a collection $S_{i_1}, S_{i_2}, \dots , S_{i_n}$, $n \leq K$, then we can choose those students numbered $i_1, i_2, \dots , i_n$ to form a student union, whose opinions will definitely cover all the opinions. \\
		$\impliedby$: \\
		If there exists a student union $S' \subseteq S$ with $|S'|\leq K$, suppose their numbers are $i_1, i_2, \dots , i_n$, $n \leq K$, then the collection $S_{i_1}, S_{i_2}, \dots , S_{i_n}$ forms a set cover. \\
		Hence $\emph{SET COVER} \leq _P \emph{ELECTION}$. \emph{ELECTION} problem is NP-complete.
	\end{proof}

	\item Not-All-Equal Satisfiability (NAE-SAT) is an extension of SAT where every clause has at least one true literal and at least one false one. NAE-$3$-SAT is the special case where each clause has exactly $3$ literals. Prove that NAE-$3$-SAT is NP-complete. (Hint : reduce $3$-SAT to NAE-$k$-SAT for some $k > 3$ at first)
	\begin{proof}
        First, NAE-3-SAT is NP, since we can find a poly-time certifier. What the certifier does is to check whether every clause has at least one true literal and at least one false one. Next, we'll establish its NP-completeness by proving 3-SAT $\leq _P$ NAE-3-SAT. However, we'll prove 3-SAT $\leq _P$ NAE-4-SAT first as an intermediary. \\
        Instance of 3-SAT: literals $x_1, x_2, \cdots , x_n$; clauses, for example, $(x_i \vee x_j \vee x_k)$ \\
        Instance of NAE-4-SAT: literals $z, y_1, y_2, \cdots , y_n$, which follow the rule that $y_i = \bar{z}$ or $y_i = z$ if $x_i$ is true or false respectively; clauses, $(z, y_i, y_j, y_k)$ corresponding to $(x_i \vee x_j \vee x_k)$, similar for negations  \\
        Now we need to prove that the instance of 3-SAT is satisfiable iff the instance of NAE-4-SAT is satisfiable. \\
        $\implies$: \\
        If there exists a truth assignment for the instance of 3-SAT, then in each clause there has to be at least one literal $x_i$ (or its negation $\bar{x_i}$) that is true. According to the mapping rule above, there has to be at least one literal $y_i$ (or its negation $\bar{y_i}$) that is equal to $\bar{z}$ in each clause, i.e., every clause is true and the instance of NAE-4-SAT is satisfiable. \\
        $\impliedby$: \\
        If there exists a truth assignment for the instance of NAE-4-SAT, then in each clause there has to be at least one literal $y_i$ (or its negation $\bar{y_i}$) that is equal to $\bar{z}$. According to the mapping rule above, there has to be at least one literal $x_i$ (or its negation $\bar{x_i}$) that is true in each clause, i.e., every clause is true and the instance of 3-SAT is satisfiable. \\
        Hence we've shown that 3-SAT $\leq _P$ NAE-4-SAT. \\
        To reduce NAE-4-SAT to NAE-3-SAT, we introduce a variable $w$ to break each 4-variable clause into two 3-variable ones, i.e., $(z, y_i, y_j, y_k) = (y_i, y_j, w) \wedge (\bar{w}, y_k, z)$. If $y_i = y_j = y_k = z$, both sides are false. If not, we can set the value of $w$ so that both sides are true. \\
        Therefore, 3-SAT $\leq _P$ NAE-3-SAT. NAE-3-SAT is NP-complete.
    \end{proof}

	\item In the Lab10, we have introduced Minimum Constraint Data Retrieval Problem (MCDR). Prove that MCDR (Version $1$ or $2$) is NP-complete. (Hint : reduce from VERTEX-COVER or $3$-SAT) \\
	\textbf{MCDR problem:} \\
	Input: 
	\begin{itemize}
		\item
		Data item set $D = \{d_1, d_2,\cdots, d_k\}$, each $d_i$ with length $l_i$ as time units
		\item
		Broadcasting channel set $\mathbf{C}=\{C_1, C_2, \cdots, C_n\}$, each $C_i$ with data sequence $S_i$ to be repeatedly broadcasted
		\item
		Broadcasting starting time $t_i$ for each $C_i$, which is defined to obtain the time differences between two channels
		\item
		The data item set $D_q \subseteq D$, which is needed by the client
	\end{itemize}

	Output (version 1):
	\[
	ans = 
	\begin{cases}
	1, \quad & \text{if } P \text{ is true} \\
	0, \quad & \text{if } P \text{ is false}
	\end{cases}
	\]
	where $P$: There exists a legal sequence of $(C_k, d_k)$ which achieves a switch cost less than $h_0$ (some specific value).
	
	\begin{proof}
        First, MCDR is NP since we can find a poly-time certifier, which checks whether the switch cost of the sequence is less than $h_0$. \\
        Next, we'll establish its NP-completeness by proving that VERTEX-COVER $\leq _P$ MCDR. \\
        Instance of VERTEX-COVER: A graph $G = (V, E)$; an integer $k$ \\
        Instance of MCDR: A data item set $D_q \subseteq D$, which is needed by the client, each element corresponds to an edge in $E$; a broadcasting channel set $\mathbf{C}=\{C_1, C_2, \cdots, C_n\}$, each $C_i$ corresponds to a vertex in $V$ (channel $C_i$ has data $d_j$ if $C_i$'s corresponding vertex is at an end of $d_j$'s corresponding edge; the sequence of data broadcast in $C_i$ is arbitrary); an integer $h = k$ \\
        We need to prove that there exists a vertex-cover of size $\leq k$ iff there exists a channel sequence of size $\leq h$ that covers the data in $D_q$. \\
        $\implies$: \\
        If there exists a vertex-cover of size $\leq k$, denoted as $S$, then we can choose the corresponding channels in $\mathbf{C}$ to form a sequence of size $\leq h$ that covers the data in $D_q$. \\
        $\impliedby$: \\
        If there exists a channel sequence of size $\leq h$ that covers the data in $D_q$, then we can choose the corresponding vertices as a vertex-cover. \\
        Hence we've reduced VERTEX-COVER to MCDR. We can conclude that MCDR is NP-complete.
    \end{proof}

\end{enumerate}

\textbf{Remark:} Please include your .pdf, .tex files for uploading with standard file names.




%========================================================================
\end{document}
