\documentclass[12pt,a4paper]{article}
\usepackage{ctex}
\usepackage{amsmath,amscd,amsbsy,amssymb,latexsym,url,bm,amsthm}
\usepackage{epsfig,graphicx,subfigure}
\usepackage{enumitem,balance}
\usepackage{wrapfig}
\usepackage{mathrsfs,euscript}
\usepackage[usenames]{xcolor}
\usepackage{hyperref}
\usepackage[vlined,ruled,linesnumbered]{algorithm2e}
\hypersetup{colorlinks=true,linkcolor=black}

\newtheorem{theorem}{Theorem}
\newtheorem{lemma}[theorem]{Lemma}
\newtheorem{proposition}[theorem]{Proposition}
\newtheorem{corollary}[theorem]{Corollary}
\newtheorem{exercise}{Exercise}
\newtheorem*{solution}{Solution}
\newtheorem{definition}{Definition}
\theoremstyle{definition}

\renewcommand{\thefootnote}{\fnsymbol{footnote}}

\newcommand{\postscript}[2]
 {\setlength{\epsfxsize}{#2\hsize}
  \centerline{\epsfbox{#1}}}

\renewcommand{\baselinestretch}{1.0}

\setlength{\oddsidemargin}{-0.365in}
\setlength{\evensidemargin}{-0.365in}
\setlength{\topmargin}{-0.3in}
\setlength{\headheight}{0in}
\setlength{\headsep}{0in}
\setlength{\textheight}{10.1in}
\setlength{\textwidth}{7in}
\makeatletter \renewenvironment{proof}[1][Proof] {\par\pushQED{\qed}\normalfont\topsep6\p@\@plus6\p@\relax\trivlist\item[\hskip\labelsep\bfseries#1\@addpunct{.}]\ignorespaces}{\popQED\endtrivlist\@endpefalse} \makeatother
\makeatletter
\renewenvironment{solution}[1][Solution] {\par\pushQED{\qed}\normalfont\topsep6\p@\@plus6\p@\relax\trivlist\item[\hskip\labelsep\bfseries#1\@addpunct{.}]\ignorespaces}{\popQED\endtrivlist\@endpefalse} \makeatother

\begin{document}
	\noindent
	
	%========================================================================
	\noindent\framebox[\linewidth]{\shortstack[c]{
	\Large{\textbf{Lab00-Proof}}\vspace{1mm}\\
	CS214-Algorithm and Complexity, Xiaofeng Gao, Spring 2020.}}
	\begin{center}
		\footnotesize{\color{red}$*$ If there is any problem, please contact TA Yiming Liu.}
		
		% Please write down your name, student id and email.
		\footnotesize{\color{blue}$*$ Name: Futao Wei  \quad Student ID: 518021910750 \quad Email: weifutao@sjtu.edu.cn}
	\end{center}
	
	\begin{enumerate}
	    \item
	    Prove that for any integer $n>2$, there is a prime $p$ satisfying $n<p<n!$. {\color{blue}(Hint: consider a prime factor $p$ of $n!-1$ and prove by contradiction)}
	    \begin{proof}
			\hfill \break
			Assume for the sake of contradiction that there exists an integer $k > 2$, such that there does not exist a prime $p$ satisfying $k < p < k!$. \\
			In other words, for any integer $n$($k + 1 \leq n \leq k! - 1$), $n$ is a composite number. \\
			For any integer $x$($2 \leq x \leq k$), $x \mid k!$. Therefore, $x \nmid k! - 1$. \\
			Since $k! - 1$ is a composite number, there must exist a prime $p$($k + 1 \leq p < k! - 1$), such that $p \mid k! - 1$, which contradicts our assumption that ``For any integer $n$($k + 1 \leq n \leq k! - 1$), $n$ is a composite number." 
	    	
        \end{proof}
	
	    \item
	    Use the minimal counterexample principle to prove that for any integer $n>17$, there exist integers $i_n\ge 0$ and $j_n\ge 0$, such that $n = i_n \times 4 + j_n \times 7$.
	    \begin{proof}
	    	\hfill \break
	    	If $P(n)$ is not true for every $n > 17$, then there are values of $n$ for which $P(n)$ is false, and there must be a smallest such value, say $n = k$. \\
	    	Since $18 = 1 \times 4 + 2 \times 7$,  $19 = 3 \times 4 + 1 \times 7$, $20 = 5 \times 4 + 0 \times 7$, $21 = 0 \times 4 + 3 \times 7$,we have $k \ge 22$ and $k - 4 \ge 18$. \\
	    	Since $k$ is the smallest value for which $P(k)$ is false, $P(k - 4)$ is true. Thus there exist integers $i_{k - 4}\ge 0$ and $j_{k - 4}\ge 0$, such that $k - 4 = i_{k - 4} \times 4 + j_{k - 4} \times 7$. \\
	    	However, we have 
	    	\begin{align*}
	    		k &= (k - 4) + 4\\
	    		&= (i_{k - 1} \times 4 + j_{k - 1} \times 7) + 4\\
	    		&= (i_{k - 1} + 1) \times 4 + j_{k - 1} \times 7
	    	\end{align*}
	    	We have derived a contradiction, which allows us to conclude our original assumption is false. 
	    \end{proof}
    
	    \item
	    Let $P=\{p_1, p_2, \cdots\}$ the set of all primes. Suppose that $\{p_i\}$ is monotonically    increasing, i.e., $p_1=2$, $p_2=3$, $p_3=5$, $\cdots$. Please prove: $p_n<2^{2^n}$. {\color{blue}(Hint: $p_i \nmid (1+\prod_{j=1}^n p_j), i=1,2,\cdots,n$.)}
	    \begin{proof}
	    	\hfill \break
	    	Define $P(n)$ be the statement that ``$p_n < 2^{2^n}$". \\
	    	\textbf{Basis step.} $P(1)$ is true, for $2 < 2^{2^1} = 4$. \\
	    	\textbf{Induction hypothesis.} For $k \geq 1$ and $1 \leq n \leq k$, $P(n)$ is true, i.e., $p_n < 2^{2^n}$. \\
	    	\textbf{Proof of induction step.} Let's prove $P(k + 1)$, i.e., $p_{k + 1} < 2^{2^{k + 1}}$. \\
	    	Since 
	    	\begin{align*}
	    		p_i \nmid (1+\prod_{j=1}^k p_j), i=1,2,\cdots,k
	    	\end{align*}
	    	We have 
    		\begin{align*}
				p_{k + 1} \leq 1+\prod_{j=1}^k p_j < 1 + \prod_{j = 1}^k 2^{2^j} &= 1+ 2^{\sum_{j = 1}^{k} 2^j} \\
				&= 1 + 2^{2^{k + 1} - 2} \\
				&< 2^{2^{k + 1}}
    		\end{align*}
    		Thus $P(k + 1)$ is true. 
	    \end{proof}
	
	    \item
	    Prove that a plane divided by $n$ lines can be colored with only $2$ colors, and the adjacent regions have different colors.
	    \begin{proof}
	    	\hfill \break
	    	Define $P(n)$ be the statement ``A plane divided by $n$ lines can be colored with only $2$ colors, and the adjacent regions have different colors." \\
	    	\textbf{Basis step.} $P(0)$ is true. \\
	    	\textbf{Induction hypothesis.} For $k \geq 0$, $P(k)$ is true. \\
	    	\textbf{Proof of induction step.} Let's prove $P(k + 1)$. \\
	    	We add another line onto the plane where $k$ lines already exist and all the regions have been colored properly. Then we recolor all the regions on one side of the new line with the opposite color. \\
	    	In this way, no color conflict forms within either side of the new line. Meanwhile, conflicts on the newly formed border are settled. 
	    	Thus $P(k + 1)$ is true. 
	    \end{proof}
	
	\end{enumerate}
	
	\vspace{20pt}
	
	\textbf{Remark:} You need to include your .pdf and .tex files in your uploaded .rar or .zip file.
	
	%========================================================================
\end{document}
